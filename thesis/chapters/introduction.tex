\chapter{Introduction} 
\label{ch:intro}
In this chapter we describe some of the initial ideas and motivations for this topic. A first guess at the possible results is made. Lastly we briefly describe an overview of the structure the thesis will have.

\section{Motivation}
\label{sec:intro:motivation}
The knowledge gained through the proposed genetic algorithm, by testing as many ideas for iterations as possible in the rather special environment of the game Cards, Universe \& Everything (\ac(CUE)) might aid during the development of a different algorithm for a similar or entirely different purpose. It's possible that games and the genre of Collectible Card Games (\ac{CCG})'s in particular might start to experiment with more unique play styles for their bots, which in turn accelerates the change of Meta (the currently most established and used strategies), because new counters could potentially be uncovered quicker and help the communities evolve.

\section{Goals}
\label{sec:intro:goals}
% My goal is to create a genetic algorithm that can create interesting and varied, but also reasonably strong decks.
The goal is to answer the question if continued evolution or a completely fresh start with every rule change is advantageous. For this purpose a genetic algorithm for deck building has to be developed. A secondary goal is to tune the algorithm for creating interesting and varied, but also reasonably strong decks. I do not expect to compete with good human built decks, as some of the limitations we will have to employ due to time constraints will likely make it difficult to compare our agents to actual human decision making. \\
Instead we're hoping to answer the question of how changing conditions effect the genetic deck building. Such big and common rule changes are fairly unique to CUE with it’s weekly energy and arena parameters, but it also has implications for other games and Collectible Card Games in particular, as for example new card releases have a comparable, if much weaker, effect.

\section{Hypothesis}
\label{sec:intro:hypothesis}
Our hypothesis is, that it will be equally successful or potentially even better to start from fresh, when the condition changes are sufficiently noticeable. Though “Hyper-mutation” may be a solution for overspecialization on a local maximum, due to loss of genetic diversity. \\
We also suspect that there might not be significant changes in performance, when using different agents, trying “combo affinity” or even simplifications of the game. Based on previous experience with genetic algorithms we assume we’ll find at first glance inconspicuous parameters that influence the result greatly. I believe that the most likely conclusion will be to keep the algorithm as simple as possible for efficiency and run it freshly after every rule change to avoid fluctuations in deck quality.

\section{Structure}
\label{sec:intro:structure}
\begin{itemize}
	\item Some background information on \ac{CCG}'s, CUE and genetic algorithms
	\item Description of the different elements of the re-implementation of the game
	\item All aspects of the algorithm and it's parameters are described and some quickly rejected
	\item A plan, based on a set of research questions, is presented for the experiment procedure
	\item The resulting data is compromised, shown and described
	\item Discussion of results, especially answering the research questions
	\item Conclusion and possible further directions to study
\end{itemize}
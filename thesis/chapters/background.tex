\chapter{Background}
\label{ch:background}
In this chapter we describe the general structure and individual components of \ac{CCG}'s and also the unique features of \ac{CUE}, the game we chose are and . Additionally some background is given into related work and some potential difficulties discussed.

\section{Collectible Card Games in general}
\label{sec:background:ccg}
Matches of a \ac{CCG} game are played by two people, each using a set of cards (their deck). The game consists of turns which generally include drawing cards to your hand and playing some of these cards. The cards usually have some kind of cost to play, a strength attribute and often some kind of effect. One example for an effect is lowering your opponents health points directly. To win the game you usually have to lower your opponents health points to zero. 
Another, often just as important or even more important part of the game, is selecting your deck. Almost all \ac{CCG}’s have a huge number of available cards to choose from. This is an example of a combinatorial optimization problem, where an exhaustive search is not feasible, so players often use shortcuts like building their deck around some kind of specific mechanic.

\section{Related work}
\label{sec:background:related}
There is a fair amount of recent related work in the area of autonomous deck building and similar optimization problems, as is reviewed well in [1]. Most research concerning CCG deck building use a specific example, since every game is slightly different and might need different strategies, though not all, for example [4]. Commonly used, because of their popularity are “Magic: The Gathering” [1] and “Hearthstone” [2]. More recently released “Legends of Code and Magic” also has some literature [3], because it is designed to be machine playable.

\section{CUE as a specific example of a CCG}
\label{sec:background:cue}
In this chapter I want to justify my decision to use a fairly unknown example of a \ac{CCG}, by explaining it’s mechanics and weekly rule changes. Since I’m especially interested in the effects of these changes and whether just mutating the decks continuously, akin to something a real player may choose to do, might be a viable option, I believe it is the right choice for me. Additionally its comparatively less complex design may allow for a fairly comprehensive representation of the game, as opposed to a game like Magic, which has around 50,000 printed cards and has even been shown to be Turing-Complete. [6] \\
The CUE developer team "Avid Games" also kindly provided us with a .tsv list of all cards including information and the effect, as written on the cards in natural language. \ref{fig:cue_cards_ex} shows an example of a few rows in this list. In total there are 2944 rows, each one representing a card, most of which with some kind of effect. \\
